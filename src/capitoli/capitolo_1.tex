\chapter{Introduzione}

Negli ultimi anni il ruolo degli informatici è decisamente cambiato. Per anni l'informatico è stato chiuso in uno sgabuzzino ad interagire in solitudine con una tastiera, bevendo bibite gassate mentre qualcuno gli diceva che cosa serviva (o almeno era convinto di sapere cosa servisse) all'azienda per crescere.

Poi sono arrivati Internet, gli smartphone, la banda larga e l'Internet delle cose (IOT) e tutto è cambiato. 

Oggi gli informatici sono diventati le nuove star! 

Il mondo è ormai tecnocratico e le nuove soluzioni informatiche e tecnologiche hanno la capacità di mutare la vita delle persone e gli andamenti dell'economia in tempi così veloci da far rabbrividire.

Tuttavia, l'informatico è rimasto (spesso) in termini di attitudine e di bagaglio culturale lo stesso di prima. Gli abbiamo fatti uscire dagli sgabuzzini, gli abbiamo messo una giacca sopra la maglietta di Star Wars e gli abbiamo spediti sui palchi dei tecnoeventi a fare pitch.

E' chiaro che le competenze tecniche siano il bagaglio fondamentale per un informatico, ma in un epoca in cui le scelte degli informatici hanno la potenzialità di cambiare la vita delle persone non si può più prescindere da far capire agli informatici che cosa accade ad una persona ``normale'' (un non informatico, un babbano) quando interagisce con un software o con un sistema tecnologico in generale. 

Per troppi anni gli informatici hanno potuto limitarsi a sviluppare per i loro simili o al massimo per i loro capi. Ora che il frutto del lavoro degli scienziati dell'informazione è destinato alle masse è arrivato il momento che gli informatici studino anche i principi fondamentali dell'interazione uomo-macchina e uomo-computer.

Questo corso è una trattazione adattata per informatici delle teorie di human computer (HCI) e human machine interaction (HMI) [interazione uomo-computer e interazione uomo-macchina]. 

Questo corso è ispirato alla teoria dell'interazione del Prof. Donald Norman ed in particolare, queste dispense sono in parte un adattamento didattico del libro ``La caffettiera del masochista'' di Donald Norman, pubblicato in Italia da Giunti e disponibile in lingua originale come ``The design of everyday things, D. Norman''. Consiglio di acquistare il libro per avere una trattazione con taglio più narrativo e sicuramente più esteso ed approfondito di quanto qui riportato.

Nel corso verranno trattati anche aspetti relativi all'Internet delle cose e alle interazioni con robot e altri sistemi ``smart''. Questi aspetti legati all'interazione con oggetti smart sono anch'essi ispirati agli studi di Norman e sono ampiamente trattati nel libro "Il Computer Invisibile, D. Norman", pubblicato in Italia da Apogeo.

L'obbiettivo di questo corso è quindi quello di fornire agli studenti del corso di laurea in Informatica gli strumenti necessari a comprendere e gestire il processo di sviluppo delle interfacce e dei prodotti interattivi. Questo corso ambisce quindi a spostare l'informatico dal suo tipico ruolo di sviluppatore per farlo diventare un progettista non solo del ``codice'' ma del prodotto nel suo insieme.

Nel corso parlerò spesso di ``prodotto'', ``business'', ``acquisto'' e altri termini legati al mondo della vendita, dell'economia e del mercato. Questo perchè l'informatico deve a mio parere sviluppare una consapevolezza fondamentale per il suo lavoro:\\

\textbf{un prodotto che nessuno compra è un prodotto inutile}. \\

Non importa quanto geniale sia il codice che avete scritto o rivoluzionario il sistema che avete implementato, se non vi curerete di far si che questo artefatto venga apprezzato e quindi utilizzato dalle persone la vostra creazione, geniale o stupida che sia, morirà dentro il vostro computer.


\textit{Queste dispense derivano dagli appunti di Simone Pepi e Francesco Iannelli pubblicati su \url{https://github.com/unipi-notes/HMI_Notes-2019-20} e relativi all' A.A 19/20.}\\

\textbf{Queste dispense sono ad esclusivo uso degli studenti del corso di Programmazione di Interfacce dell'Università di Pisa. E' vietata la divulgazione, copia o riproduzione in qualsiasi forma.}
